\documentclass[UTF8, 12pt, a4paper]{ctexart}
\newcommand{\ud}{\mathop{}\negthinspace\mathrm{d}}
\usepackage{CJK}
\usepackage[top=1in,bottom=1in,left=1.25in,right=1.25in]{geometry}
%\geometry{left=2.54cm, right=2.54cm, top=3.18cm, bottom=3.18cm}
%\newtheorem{theorem}{定理}[section]
\usepackage{graphicx}
\usepackage{listings} %insert code 
\usepackage{amsmath, amsthm, amssymb, graphicx}
\usepackage[bookmarks=true]{hyperref}
\title{读书报告}
\author{姓名:杨俊鹏{} 学号:2020302191933}
%\date{\today}
\begin{document}
\maketitle
\section{第一章{} 概述}
\subsection{电网、电力系统和动力系统的划分}
电网:由各类升降压变电站(换流站)、各种电压等级的交直流输电线路所组成的整体。作用是输送、控制和分配电能。

电力系统:由发电机、升降压变压器(换流站)、各种电压等级的交直流输电线路和广大用户的用电设备所组成的完成电能生产、输送、分配及消费的统一整体。由发电机、发电厂、输电、变电、配电以及负荷组成的系统。

动力系统:由带动发电机转动的动力部分、发电机、升压变电站、输电线路、降压变电站和负荷等环节构成的整体。由带动发电机转动的动力部分与电力系统构成的整体。
\subsection{电力系统和电网}
电力系统:由发电机、电力网和负荷所组成的统一整体。

发电机:电能生产,一次能源转换成电能;
电力网:电能输送和分配,包括输电网和配电网;
负荷(电能消耗Load):将电能转换成其他形式能量的用户用电设备,电动机/照明/电炉等

电力系统中除发电厂及用户以外的部分即为电网,包括输电网和配电网。

输电网:将各个发电厂通过较高电压(如220kV、330kV、500kV、750kV、1000kV等)的线路相互连接,使所有同步发电机之间并列运行,同时将发电厂发出的电能输送到负荷中心。

配电网:电能输送到负荷中心后,用较低电压(110kV、35kV、10kV等)的线路供给电力用户,起到分配电能的作用。

总装机容量:指该系统中实际安装的发电机组额定有功功率的总和,以kW为单位计。

年发电量:指该系统中所有发电机组全年实际发出电能的总和,以kWh为单位计。

最大负荷:指规定时间内,电力系统总有功功率负荷的最大值,以MW为单位计。

年用电量:系统中所用用户全年所用电量的总和,以kWh为单位计。

最高电压等级:电力系统中最高电压等级的电力线路的额定电压,以kV为单位计。

一次设备:直接用于生产和使用电能的高压电气设备(包括发电机、变压器、输电线路和用电设备)。

一次系统:由一次设备相互连接,构成发电、输电、配电或直接用于生产的电气主回路,又称为一次回路或一次接线系统。

二次设备:对一次设备进行监察、测量、控制、保护和调节的辅助设备,称为二次设备。不直接和电能产生联系。

二次系统:由二次设备相互连接,构成对一次设备进行监测、控制、调节和保护的电气回路称为二次回路或二次接线系统。

电力网电压等级,相电压、线电压说明:系统额定电压是线电压,功率通常指三相功率,线路参数为单相参数。

电网电压等级:

我国规定的电力网额定电压等级(三相交流系统的线电压):3,6,10,20,35,63,110,220,330,500,750kV,1000kV。

直流输电的单级电压:±400,±500,±660,±800kV

现在我国的特高压电网最高电压等级:

交流:最高线路电压1000kV、最远输送距离大于1000km,直流:最高电压±1100kV、输电距离大于1000km

电气设备的额定电压:

为使电气设备有良好的运行性能,国家标准规定各级电网电压在用户处的电压偏差不能超过±5%;允许线路首端电压比额定电压高5%,线路末端电压比额定电压低5%。

确定额定电压(原则)

线路(电网)额定电压=用电设备额定电压

发电机额定电压=1.05线路额定电压

发电机的额定电压:由于发电机总是在线路的首端,所以它的额定电压应比电网额定电压高5%,用于补偿电网的电压损失

当降压变压器二次侧输电距离较短,或变压器阻抗较小(Us%小于7%)时,二次绕组的额定电压可只比同级电网的额定电压高5%
\subsection{电力系统的运行}
电力系统运行的特点:
1.电能的生产和使用是同时完成的,所以电能难以储藏是电能生产的最大特点。
2.正常输电过程和故障过程都非常迅速。
3.具有较强的地区性特点。
4.与国民经济各部门关系密切。

基本要求可以简单概括为:安全、可靠、优质、经济、环保

保证供电的安全可靠:电力用户分为三类,当系统发生事故,出现供电不足情况时,应首先切除三类用户的用电负荷,以保证一类、二类用户的用电。

保证电能的良好质量:电能质量指标:频率、电压和波形

1.系统频率:取决于系统有功功率平衡。
2.节点电压:取决于系统无功功率平衡。
3.波形质量:是由谐波污染引起。

我国规定的电力系统的额定频率为50Hz,大容量系统允许频率偏差±0.2Hz,中小容量系统允许频率偏差±0.5Hz。35kV及以上的线路额定电压允许偏差±5%;10kV线路额定电压允许偏差±7%;电压总谐波畸变率不大于4%,380V/220V线路额定电压允许偏差±7%,电压波形总崎变率不大于5%。
\subsection{电力工业发展趋势}
一、节能减排、大力开发新能源、走绿色电力之路

二、建设以特高压为骨干网架的坚强电网

三、组成联合电力系统

四、加强智能电网的建设
\newpage{}
\section{第二章{} 电力系统的负荷}
用电设备所消耗电功率的总和称为电力系统综合负荷,简称为电力系统负荷或负荷。负荷加上电网的功率损耗称为电力系统的供电负荷。供电负荷与发电厂的厂用电之和称为电力系统的发电负荷;发电负荷包括厂用电、电力网的功率损耗(网损)和综合负荷;供电负荷包括网络中的功率损耗和综合负荷,而不包括厂用电。负荷不是恒定值,是随时间或者运行电气量(电压、频率)变化的变量。

有功功率 (P)
把电能转换为其他能量(机械能、光能、热能等)并在用电设备中真实消耗掉的功率。
无功功率 (Q) 并不做功,只是用来完成电磁能量的相互交换所需的功率。与有功功率同样重要。
\subsection{负荷的表示方法}
负荷功率的表示方法:电力系统的负荷功率通常用复数功率或电流表示

负荷曲线:描述在某一段时间内用电负荷大小随时间变化规律的曲线。

日负荷曲线:描述一天24小时内负荷变化情况的曲线。

日负荷率:$k_m=\frac{P_{av}}{P_{max}}$ \qquad 最小负荷系数:$\alpha=\frac{P_{min}}{P_{max}}$

$k_m$、$\alpha$值愈小,表明负荷波动愈大,发电机的利用率愈差。采用“削峰填谷”等措施,尽量使得$k_m$、$\alpha$趋近于1。

年负荷曲线 分为年最大负荷和年持续负荷曲线。

年最大负荷曲线:描述一年内每月(或每日)最大有功负荷随时间变化情况的曲线。

最大负荷利用时间$T_{max}$:若系统始终以最大负荷运行,经过一段时间后其围成的面积恰好等于曲线所围成的面积,即等于全年的电能消耗量时,则称这一段时间为最大负荷利用时间。
\subsection{负荷特性与模型}
电力系统负荷的运行特性广义地可分两大类,即负荷随时间而变化的规律(负荷曲线)和负荷随电压或频率而变化的规律(负荷特性)。

负荷特性主要研究方法——实测法、辨识法。
通常负荷特性表示的是从主干功率传递点(通常为母线)看出的综合负荷特性。

负荷特性按功率分为有功负荷特性和无功负荷特性,按性质分为静态特性和动态特性。
静态特性:反映电压和频率缓慢变化时负荷功率的变化特性。
动态特性:反映电压和频率急剧变化时负荷功率的变化特性。

负荷静态特性定义:反映负荷功率随系统运行参数(电压或频率)的变化而变化的规律或关系。

1负荷静态电压特性:当频率维持额定值不变时,负荷功率与电压的关系。

2负荷静态频率特性:当负荷端电压维持额定值不变时,负荷功率与频率的关系。

某电力系统的综合负荷静态特性:
静态电压特性的特点:U↓,负荷的P↓,Q↓;
频率特性曲线的特点:f↓,负荷的P↓,Q↑。

负荷静态特性常用近似模型

1.多项式负荷静态特性
\[P=P_N[A_P(\frac{U}{U_N})^2+B_P(\frac{U}{U_N})+C_P][1+\frac{\mathrm{d}(\frac{P}{P_N})}{\mathrm{d}(\frac{f}{f_N})}\Bigl{|}_{f_N}(\frac{\Delta f}{f_N})]\]
\[Q=Q_N[A_Q(\frac{U}{U_N})^2+B_Q(\frac{U}{U_N})+C_Q][1+\frac{\mathrm{d}(\frac{Q}{Q_N})}{\mathrm{d}(\frac{f}{f_N})}\Bigl{|}_{f_N}(\frac{\Delta f}{f_N})]\]

第一个方括号内的各项表现了负荷的电压特性,其中的第一项为等效恒定阻抗负荷,第二项为等效恒定电流负荷,第三项为等效恒定功率负荷;第二个方括号内反映的是负荷的频率特性

2.幂函数式负荷静态特性
\[P=P_N(\frac{U}{U_N})^{P_U}(\frac{f}{f_N})^{P_f} \qquad Q=Q_N(\frac{U}{U_N})^{Q_U}(\frac{f}{f_N})^{Q_f}\]
\[
    P_U = \frac{\mathrm{d}{\frac{P}{P_N}}}{\mathrm{d}{\frac{U}{U_N}}}\Bigl{|}_{f_N}\qquad
    P_f = \frac{\mathrm{d}{\frac{P}{P_N}}}{\mathrm{d}{\frac{f}{f_N}}}\Bigl{|}_{f_N}
\]
\[
    Q_U = \frac{\mathrm{d}{\frac{Q}{Q_N}}}{\mathrm{d}{\frac{U}{U_N}}}\Bigl{|}_{f_N}\qquad
    Q_f = \frac{\mathrm{d}{\frac{Q}{Q_N}}}{\mathrm{d}{\frac{f}{f_N}}}\Bigl{|}_{f_N}
\]

电压特性系数:$P_U$,$Q_U$ 分别为负荷有功功率和无功功率的电压特性系数;物理含义:负荷功率和频率均为额定值时,功率对电压的变化率

频率特性系数:$P_f$,$Q_f$ 分别为负荷有功功率和无功功率的频率特性系数;物理含义:负荷功率和电压均为额定值时,功率对频率的变化率

负荷幂函数式中的幂系数就是负荷的特性系数,较多项式中的各系数容易确定,因而负荷特性常用幂函数形式表示。

3.恒定阻抗式负荷静态特性
\[\begin{cases}
    P=\frac{U^2}{R^2+(2\pi fL)^2}R\\
    Q=\frac{U^2}{R^2+(2\pi fL)^2}X
\end{cases}\]

模型简单,可以提高计算速度。误差较大,只在负荷容量小、端电压波动不大、精确度要求不高的情况下使用。

负荷静态模型的使用范围:电力系统的潮流计算频率稳定、电压稳定分析计算,无功优化补偿等分析计算

负荷动态特性

分析:一般用微分方程和代数方程组成的方程组表示。

数学模型定义:负荷在系统电压和频率快速变化时,应计及其动态特性,并用一个微分方程描写,称之为负荷动态模型。
\[P=\varphi_p(U,f,\frac{\mathrm{d}U}{\mathrm{d}t},\frac{\mathrm{d}f}{\mathrm{d}t},\frac{\mathrm{d}U}{\mathrm{d}f},\cdot\cdot\cdot\cdot\cdot\cdot)\]
\[Q=\varphi_p(U,f,\frac{\mathrm{d}U}{\mathrm{d}t},\frac{\mathrm{d}f}{\mathrm{d}t},\frac{\mathrm{d}U}{\mathrm{d}f},\cdot\cdot\cdot\cdot\cdot\cdot{})\]

电力系统综合负荷主要成分是异步电动机;异步电动机的动态特性比静态特性复杂得多:通常分为机械暂态过程(只考虑转子绕组机械暂态)、机电暂态过程(进一步考虑转子绕组电磁暂态)、电磁暂态过程(更进一步考虑定子绕组和网络的电磁暂态)

负荷动态特性模型的建模思路

综合负荷动态模型:
1.静态负荷模型。
2.异步电动机负荷模型。

负荷的等值感应电动机参数可采用经验参数,或进行实测和估计。
\subsection{电力系统中的谐波}
本课程中,对负荷模型一般都作简化处理。

潮流计算中,负荷常用恒定功率(P、Q恒定)表示只有在对计算精度要求较高时,才考虑负荷的静态特性。

短路计算中,负荷或表示为含源阻抗支路,或表示为恒定阻抗支路。
\newpage{}
\section{第三章{} 电力系统主设备元件}
\subsection{电力变压器的等值电路及参数计算}
(1)双绕组变压器:Γ型等值电路的励磁支路通常放在功率输入侧(电源侧)

参数:任何一台变压器出厂时,制造厂家都会在变压器的铭牌上或出厂试验书上给出代表其电气特性的四个参数,即短路损耗(也称负载损耗)$\Delta P_k(\Delta P_s)$、短路电压(也称短路阻抗)百分数Us\%、空载损耗$\Delta P_0$、空载电流百分数$I_0$\%。前两个参数由短路试验得出,后两个参数由空载试验得出。根据以上四个电气特性数据,即可计算出等效电路中的:
\[R_T=\frac{\Delta P_sU^2_N}{S_N^2}\times{10}^3(\Omega)\]
\[X_T=\frac{U_s\%}{100}\times\frac{U_N^2}{S_N}\times{10}^3(\Omega)\]
\[G_T=\frac{\Delta P_0}{U^2_N}\times{10}^{-3}(S)\]
\[B_T=\frac{I_0\%}{100}\times\frac{S_N}{U_N^2}\times{10}^{-3}(S)\]

单位需根据题意自行变化

(2)三统组变压器

电力变压器的等效电路及参数计算:

1.空载试验求励磁导纳(励磁电导和励磁电纳)
空载实验:1侧加$U_N$,另两侧开路,导纳支路的参数计算公式与双绕组变压器相同。

2.短路试验求串联阻抗

短路试验:一侧加$I_N$,一侧短路,一侧开路,短路参数注意按容量变比进行计算。
\[\Delta P'_{S(a-b)}=(\frac{S_N}{\min{\{S_{Na},S_{Nb}}\}})^2\Delta P_{S(a-b)}\]
$$\begin{cases}
    \Delta P'_{S1-2}=\Delta P_{S1}+\Delta P_{S2}\\
    \Delta P'_{S1-3}=\Delta P_{S1}+\Delta P_{S3}\\
    \Delta P'_{S2-3}=\Delta P_{S2}+\Delta P_{S3}
\end{cases} 
$$
$$
\begin{cases}
    \Delta P'_{S1}=\frac{1}{2}(\Delta P'_{S1-2}+\Delta P'_{S1-3} - \Delta P'_{S2-3})\\
    \Delta P'_{S2}=\frac{1}{2}(\Delta P'_{S1-2}+\Delta P'_{S2-3} - \Delta P'_{S1-3})\\
    \Delta P'_{S3}=\frac{1}{2}(\Delta P'_{S1-3}+\Delta P'_{S2-3} - \Delta P'_{S1-2})
\end{cases}
$$

三绕组变压器的额定容量$S_N$:三个绕组中容量最大的一个绕组的容量。公式中的$\Delta P_S$是指绕组流过与变压器额定容量,$S_N$对应的额定电流$I_N$时所产生的有功损耗。国标规定:$U_S$\%已经规算到$S_N$相对应的值。

三绕组布置方式:

升压布置结构:中压绕组、低压绕组、高压绕组

降压布置结构:低压绕组、中压绕组、高压绕组

对于第一种排列方式来说,其高压、中压绕组间的漏磁通道较长,阻抗电压较大,而高压、低压绕组间以及中压、低压绕组间的漏磁通道较短,阻抗电压较小。所以这种排列方式适用于需将功率自低压绕组经高压、中压绕组输入系统的升压变压器,以减小低压、高压间及低压、中压间的电压损耗;也适用于功率主要自高压侧传输向低压侧的降压变压器,以减小高压、低压间的电压损耗。当低压侧各断路器的开断能力不足,需限制短路电流时,应采用第二种排列方式,以满足安全运行的要求

电抗负值通常出现在中间位置的绕组上,由于其值不大,实际计算中常做零处理。
电抗负值只反映绕组间自感、互感作用的数学等值,不反映物理性质,即负值不代表绕组电抗为容性。

变压器不论实际接法,求出参数都是等值成Y/Y接法中的单相正序参数。

(3)自耦变压器

(4)变压器的π型等值电路

特点:

1.三个参数是数学等值,无物理含义。
2.三个参数均与变比k有关。
3.两并联参数符号相反,负号出现在电压等级高侧。
4.三个阻抗之和恒为0,构成了谐振三角形。

更适合对复杂电力系统进行计算:(1)与变压器相连的元件参数无需折算。(2)计算出的电气量就是实际值,无需再折算。
\subsection{输电线路}

输电线路:分为架空导线和电缆线路。

架空线路:由导线、避雷线、杆塔、绝缘子和金具组成。

架空导线的特殊类型---分裂导线

构造:子导线+间隔棒;作用:等效地增大导线半径,防止电晕(增大导线面积),减小线路感抗,减小线路的等效电抗及增大线路的等效电容。
顺便指出,由于分裂导线能够改变输电线路参数,因此人们可以通过对分裂导线的合理布置及适当排列三相导线位置,使输电线路的参数接近或达到阻抗匹配。如此可以大大提高线路的输送功率,这就是现代紧凑型输电线路的基本原理。

(3)三相对称运行时架空输电线路的参数计算

单位长线路等值电路和参数
分布式参数:导线参数沿线路均匀分布,用单位长(/km)参数r、x、g、b表示

架空线路受气候、地理、架设方式的影响,r、x、g、b要变

电缆尺寸标准化,外界影响小,一般不变(本课程不研究)

电阻:反映线路通过电流是产生的有功功率损耗,线路通电流发热,消耗有功功率

电抗(电感):反映载流线路周围产生的磁场效应,交流电流→交变磁场(磁场效应)→感应电势(自感、互感)抵抗电流→X=ωL

电导:反映电晕现象产生的有功功率损失,线路加电压→绝缘漏电(较小),电晕损耗(一定电压下的导线周围空气放电)→有功功率损耗

电纳(电容):反映载流线路周围产生的电场效应,电场→线/线、线/大地电容(电场效应)→交变电压产生电容电流→B=ωC
\[r = \frac { \rho } { n S } (\Omega / km)
\]
\[x _ { 1 } = 0.1445 \lg {}{\frac { D _ { e q } } { r _{e q } }}+ \frac { 0.0157 } { m }(\Omega / km)
\]
\[g=0
\]
\[b = w C= \frac { 7.58 } {\lg{}\frac { D_{e q } }{ r_{eq} } }\times 10 ^ { - 6 }(S/ km)
\]

导线间的互几何均距:$D _ { eq } = \sqrt [ 3 ] { D _ { 12 } D _ { 23 } D _ { 31 } } $\quad 

分裂导线的等值半径: $r _ { e q } =  \sqrt[m]{ r \prod _ { k = 2 } ^ {m} d _ { 1k } } $

(4)电力线路的等值电路

一字型等值电路使用条件:线路长度不超过100km的架空线路及不长的电缆线路,且工作电压不高时,可忽略线路电纳和电导。

T形和π形等值电路使用条件:线路长度在100~300km之间的架空线路或长度不超过100km的电缆线路。
\subsection{高压开关电器}
开关电器:

1.断路器:既用来断开或关合正常工作电流,也用来断开负荷电流或短路电流,高压断路器是开关电器中最复杂、最重要、开断功能最完善的开关设备

2.高压熔断器:用来自动断开短路电流或负荷电流的开关电器。开断后,经过更换熔丝等检修工作后,电路才能闭合。

3.高压负荷开关:能在正常情况下开断和关合工作电流的开关电器,也可以开断负荷电流,不能开断短路电流。一般情况下负荷开关要与熔断器配合使用。

4.隔离开关:产生明显电位开断点,使电力系统中运行的各种高压电器设备与电源间形成可靠的绝缘间隔,以便对退出运行的设备进行试验和检修,保证高压设备在检修工作时的安全;主要用于检修时隔离电压或运行时进行倒闸操作的开关电器,不能开断负荷电流和短路电流,通常只用于开合有电压、无负荷的线路;可用于不产生强大电弧的某些切换操作(切断电容电流和限量的空载线路)

5.自动重合器和自动分段器:是自具保护和控制功能的配电网开关电器。

电弧是一种等离子体,交流电弧的特点:电流每半个周期要经过零值一次。在电流经过零值时,电弧会自动熄灭。电弧会发生重燃。

开关电弧的产生和熄灭

电弧产生:强电场发射---热电子发射---热游离---形成稳定的电孤孤柱

电弧熄灭:高压回路开断瞬间会产生电弧,电弧熄灭后,虽然电源已不再向电弧间隙输入热能,但弧隙中仍存在游离粒子,不能立即恢复到完全绝缘的状态。消游离方式:扩散方式和复合方式

灭弧室的工作大致可分为下列三个阶段:封闭泡阶段、气吹阶段、回油阶段。

提高断路器的熄弧能力的基本措施:

加速断口介质强度的恢复速度并提高其数值是提高开关熄弧能力的主要方法,为此可以采取以下措施:
(1)采用绝缘性能高的介质
(2)提高触头的分断速度或断口的数目,使电弧迅速拉长;(电弧拉长,实际上是使电弧上的电场强度减小,则游离减弱,有利于灭弧,伏安特性曲线拾高)
(3)采用各种结构的灭弧装置来加强电弧的冷却,以加快电流过零后弧隙的去游离过程。
\subsection{高压互感器}
互感器的任务:把高电压和大电流按比例地变换成低电压和小电流;把电力系统处于高电位的部分与处于低电位的测量仪表和继电保护部分分开,保证运行人员和设备的安全;接入电路之后,互感器的二次绕组必须有一点接地,以防一、二次侧的绝缘击穿时,高压危及人身和设备的安全。
工作原理、等值电路和变压器完全一致,但在工作特点、性能要求、结构上有较大差别。测量准确度是表征互感器性能的重要指标。绝缘方式是决定高压互感器结构形式的主要因素。

互感器分类:电压互感器和电流互感器。

电压互感器

电磁式电压互感器工作时其一次绕组跨接在所需测量的电压上,电压互感器的负载是并接在二次绕组上的仪表和继电器的电压线圈。由于这些电压线圈的阻抗很大,电压互感器在工作时二次侧接近开路,所以电压互感器实质上为一容量极小(额定容量在 1000V·A以下)的降压变压器,其一次绕组的匝数$W_1$远大于二次绕组的匝数$W_2$。

工作特点:

原边并联在电路中,副边接电压表计或电压线圈;

一次电压和额定值差别不大(一次侧电源等效于电压源);

近似空载下运行,即二次侧近似开路(容量小,电压互感器$S_N<$1000VA)

运行时,二次侧不能短路。

$U_{1n}$为电网额定电压(110kV、220kV等);$U_{2n}$统一定为100V或$100 / \sqrt{3}$;对某一电压等级,额定变比是固定的。

当有检测单项接地故障的需要时,电压互感器需设第三绕组来获取零序电压。

电流互感器

电磁式电流互感器工作时,其二次绕组是串接在线路中的,电流互感器的负载则是串联后接到二次绕组上的仪表和继电器的电流线圈。由于这些电流线圈的阻抗很小,电流互感器在工作时接近短路状态。

工作特点:

原边串在电路中,副边同电流表计或线圈相串联;

一次侧电流只取决于系统,不受互感器负载影响(一次侧电源等效于电流源);

二次绕组近似在短路状态下工作,不能开路运行;

电流互感器能在电流变化范围较大情況下保持测量的精度;

$I_{1n}$为高压电网额定电流;$I_{2n}$统一定为5A或1A;对于一次侧某一额定电流,变比是固定的。

误差原因(定性分析)

电磁式高压互感器: $\dot { U } _ { 1 } = - \dot { U } _ { 2 } ^ { \prime } + \dot { I } _ { 0 } Z _ { 1 } - \dot { I } _ { 2 } ^ { \prime }( Z _ { 1 } + Z_ { 2 } ^ { \prime } )$,降低$I_0$和$I'_2$将有助于减小电压互感器误差; 

电流互感器的误差受励磁电流$I_0$影响:$\dot{I'}_1=-\dot{I'}_2+\dot{I}_0$

电容式电压互感器
由多个电容串联而成,一端接高压导线,一端接地。实质是一个电容分压器。分压电容还可兼做高频载波通信的耦合电容;可防止因铁心饱和引起的铁磁谐振;制造简单,体积小重量轻,造价低,占地少;电压越高效果越明显,多用于110kV及以上电力系统
\newpage{}
\section{第四章{} 电力系统的接线方式}
\subsection{发电厂、变电所的主接线}
发电厂变电站的电气主接线是由发电厂变电站的所有高压电气设备通过连接线组成的、用来接受和分配电能的强电流、高电压电路,又称为电气一次接线

对电气主接线的基本要求:
1.可靠性:是电力生产的首要任务。
2.灵活性:主接线应能适应于各种工作情況和运行方式,能根据运行情況方便地退出和投入电气设备。
3.经济性:经济合理性。

电气主接线图:根据电气设备的作用和对它们的工作要求,用规定的图形符号和文字符号、按一定顺序排列,详细地表示出电气设备基本组成和连接关系的接线图,也称电气一次接线图和主接线图。通常采用单相图,根据需要局部地方绘成三相图

电气主接线直接影响到供电可靠性、电能质量、运行灵活性、配电装置布置、电气二次接线和继电保护及自动装置的配置问题,选型很重要。

主接线的基本形式分为有汇流母线和无汇流母线,
汇流母线的作用是汇集和分配电能

有汇流母线:单母线接线;双母线接线;3/2接线

单母线接线

优点:接线简单清晰、使用设备少、投资小、运行操作方便、误操作机会少、便于扩建;缺点:可靠性和灵活性较差

其他形式:单母分段、单母加旁路

双母线接线

优点:可靠性较高、调度灵活、扩建方便;缺点:接线复杂、设备多、造价高;配电装置复杂,经济性差;汇流母线故障,会引起整个装置短时停电;容易引起误操作,不便于实现自动化

其他形式:其中一组汇流母线分段、加装旁路母线(在带旁路的双母线接线中,旁路母线的作用是检修断路器不停电)

一个半断路器接线

优点:具有很高的可靠性和灵活性;缺点:使用设备多、造价高、经济性差、二次接线和继电保护整定复杂

无汇流母线:单元接线;桥形接线;多角形接线

单元接线

特点:发电机直接经变压器接入高压系统

优点:简明清晰、故障范围小、运行可靠灵活、配电装置布置简单、操作方便;缺点:单元中任一元件故障都会使整个单元停止工作

桥形接线

优点:接线简单清晰、使用电器少、造价低、容易发展成为单母或双母接线;缺点:内桥接线中主变故障时,需停相应线路;外桥接线中线路故障时,需停相应的主变压器;操作过程中隔离开关要作为操作电器用。

多角形接线

优点:供电可靠性、运行灵活性和经济性较高;缺点:当检修某一台断路器时接线变成开环运行,若恰有另一台断路器故障就会造成停电,继电保护整定计算复杂,选择设备容量过大

桥型接线中:只适用于具有两台变压器和两回路出线的较小容量的发电厂和变电站

内桥接线适用于主变压器不经常切换或线路较长、故障几率较高且要求尽量减少电网脱环运行的情况。

外桥接线适用于有穿越功率的情况、主变压器切换频繁或线路较短、故障几率较少的情况。

电气倒闸操作:通过操作隔离开关、断路器以及挂、拆接地线将电气设备从一种状态转换为另一种状态的有序操作,叫做倒闸操作。

操作顺序基本要求:隔离开关和断路器配合操作时,必须满足隔离开关“先通后断”或等电位(作为操作电器)操作原则。

例如对线路进行送电操作的顺序应该是:先合母线隔离开关QSB,再合出线隔离开关QSL,最后合断路器QF;而对出线进行停电操作时,则应先断开断路器QF,再断开线路隔两开关QSL,最后断开母线隔离开关QSB。
\subsection{中性点接地方式}
电力系统的接地方式

工作接地:电气设备因为正常工作需要而进行的接地。

保护接地:当电气设备的金属外壳、构架、线路杆塔由于绝缘损坏有可能带电时,为了防止危及人身和设备的安全而设置的接地。

防雷接地:防雷保护装置泄放雷电流入地的接地。

中性点接地属于工作接地,分为两类:大电流接地系统(直接接地系统)和小电流接地系统(非直接接地系统)

大电流接地系统:中性点直接接地,中性点经低阻接地。可靠性低,大电流,适用电压等级高

特点: 接地电流很大,必须迅速切除接地相甚至三相, 供电可靠性降低。
故障时非接地相对地电压不升高,对于系统的绝缘性能要求相应降低。

小电流接地系统:中性点经消弧线圈接地,中性点不接地,中性点经高阻接地。可靠性高,高电压,适用电压等级低

特点:在小电流接地系统中发生单相接地短路故障时,由于中性点非有效接地,不会出现电源被短接的现象,故障点不会产生大的短路电流,因此系统可以带接地故障继续运行(一般允许1-2个小时),待做好停电准备后再停电排除故障,提高供电可靠性。但是此时当系统带故障运行时,非故障相电压升为线电压,会破坏绝缘,因此线路和设备的绝缘费用增大,容易引发各种过电压,甚至导致故障严重化。

中性点经消弧线圈接地
\[\dot { I } _ { f } ^ { ( 1 ) } = ( - \dot { U } _ { C } ) [ j \omega ( C _ { 11 } + C _ { 22 } + C _ { 33 } ) -j \frac { 1 } { \omega L }]\]

将消弧线圈的L值偏离调谐的程度用脱谐度$v$表示:
$v = 1 - k= \frac { I _ { C } - I _ { L } } { I _ { C } }$$\qquad$
补偿度:
$k = \frac { I _ { L } } { I _ { C } }$

消弧线圈的补偿方式:

过补偿:
$I _ { L } > I _ { C } , k > 1,v < 0 $
$\qquad$
欠补偿:
$I _ { L } < I _ { C } , k < 1,v > 0 $

在欠补偿的情况下(电感L大,电感电流小于电容电流),如果电网有一条线路跳闸(此时电网对地自部分电容减小)时,或当线路非全相运行(此时电网一相或两相对地自部分电容减小)时,或$U_0$偶然升高使消弧线圈饱和而致电感L值自动变小时,脱谐度趋近0,消弧线圈可能趋近完全调谐,完全调谐可以使单相接地电流大为减小,从而产生严重的中性点位移,正常运行状态下中性点产生谐振过电压。因此,消弧线圈一般应采取过补偿的运行方式。

中性点接地方式的一般原则

一、110kV及以上电网:中性点直接接地;
二、10~66kV电网:中性点不接地,经消弧线圈接地,经低阻接地;
三、10kV电缆线路:经低阻接地;
四、3~6kV厂用电系统,规模较小的6~10KV配电系统:中性点经高阻接地;
五、发电机:中性点不接地,经消弧线圈接地
\newpage{}
\section{第五章{} 电力系统稳态分析}
\subsection{电力系统的潮流计算}
潮流计算是电力系统稳态分析中的一种最基本的计算,包括经典算法和计算机算法。

电力系统潮流计算的任务就是针对具体的电力网络结构,根据给定的负荷功率和电源母线电压,计算网络中各节点的电压和各支路中的功率及功率损耗。

作用:电力网规划设计;电力系统运行(稳态、短路、稳定等);继电保护、自动装置整定计算。

电力网环节的定义:电力网等值电路中通过同一个电流的阻抗支路。化繁为简,任何复杂的电力网络都可由一系列电力网环节组成。

电力网的功率损耗:电力网由线路和变压器组成,因此,电力网的功率损耗也由线路的功率损耗和变压器的功率损耗所构成

电力网环节的功率损耗

变动损耗:电流通过电力网环节支路电阻和电抗时产生的损耗;固定损耗:电压施加于并联等值导纳时产生的损耗

线路功率损耗的计算

线路固定功率损耗:对地电容支路,消耗容性无功功率。(线路为充电功率);线路变动功率损耗:环节阻抗支路,有功损耗和无功损耗。高压输电线路无功功率损耗很大(因为线路环节电抗较大)

变压器功率损耗的计算

固定损耗:导纳支路中的功率损耗,有功损耗和无功损耗;变动功率损耗:环节阻抗支路,有功和无功损耗

电网环节的功率平衡和电压平衡

电力系统电压分析计算中的三个基本概念

电压降落:电网任意两点的相量差 $\quad d \dot { V } = \dot { V } _ { 1 } - \dot { V } _ { 2 }$

电压损耗:电力网中任意两点电压的代数差 $\quad | \dot { V } _ { 1 } | - |\dot { V } _ { 2 } |= V _ { 1 } - V _ { 2 }$

电压偏移:电力网中任意点的实际电压U同该处网络额定电压Un的数值差称为电压偏移。 $\quad m \% = \frac { V - V _ { N }  } { V _ { N } }\times 100 \quad $ 电压偏移的大小,直接反映了供电电压的质量

交流电网中关于功率传送的重要概念
\[\Delta V = \frac { P R + Q X } { V } \qquad \delta V = \frac { P X - Q R } { V }\]

计算时的功率和电压应该使用同一端的数值

电力网环节中的功率传输方向:

电压降落的纵分量主要决定两点间的电压幅值差;电压降落的横分量主要决定电压相角差。

电压降落的纵分量(电压幅值差);电压降落的横分量(电压相角差)

电压的相位差主要由通过电力网环节的有功功率决定,而与无功功率几乎无关;数值差主要由通过电力网环节的无功功率决定,而与有功功率几乎无关。

在纯电抗元件中,元件两端存在电压幅值差是传送无功功率的条件;元件两端存在电压相角差是传送有功功率的条件。

感性无功功率从电压幅值较高的一端流向电压幅值较低的一端;容性无功功率从电压幅值较低的一端流向电压幅值较高的一端;有功功率从电压相位超前的一端流向电压相位落后的一端。

实际的网络元件都存在电阻:考虑电阻影响,有功功率的传输增加电压降落的纵分量(电压幅值差);考虑电阻影响,无功功率的传输减少电压降落的横分量(电压相角差)。

潮流计算迭代法适用于已知线路首端电压$U_1$和末端功率$S_{LD}$,为了提高计算精度

空载线路的法拉第效应(电容效应):线路空载时,负荷有功和无功均为零,由线路电容功率使其末端产生工频电压升高的现象。
\subsection{电力系统的频率与有功功率}
电力系统的频率特性

电力系统综合负荷的有功功率---频率静态特性

负荷频率特性:描述电力系统负荷的有功功率随频率变化的关系曲线。(在额定频率附近,该曲线近似为直线)
负荷调节效应系数的标幺值注意:其值与系统各类负荷的比重和性质有关,一般取值由试验或计算求得1~3,不能人为整定。

负荷调节效应系数(单位:$MW/Hz$): 
\[k _ { LD } =  \frac { \Delta P  } { \Delta f } \qquad k _ { LD } ^ { * } = \frac {  \Delta P  / P _ { LDN } } { \Delta f / f_{N} }=  \frac { \Delta P  ^ { * } } { \Delta f  ^ { * } }\]

发电机组的有功功率---频率静态特性

发电机组功频特性:描述发电机组输出的有功功率与频率之间关系的曲线。
发电机组的静态功频特性(调速器的作用),可用一条直线近似表示。
发电机单位调节功率可以人为整定,但其调整范围受到机组调速机构的限制。

发电机组单位调节功率的标幺值,亦称发电机组功频静态特性系数:$k_G^{*}$
\[k _ { G } ^ { * } = - \frac {  \Delta P _ { G } / P _ { GN } } { \Delta f / f_{N} }= - \frac { \Delta P _ { G  } ^ { * } } { \Delta f  ^ { * } }\]

发电机组的单位调节功率:(单位:$MW/Hz$ 或 $kW/Hz$)
\[k _ { G } = - \frac { \Delta P _ { G } } { \Delta f } = k_G^{*} \frac{P_{GN}}{f_N}\]

静态调差系数(调差率):$\delta = \frac { 1 } { k _ { G } }$
\[
    \begin{cases}

        k _ { G \Sigma } = \sum _ { i = 1 } ^ { m } k_ { Gi }\\
        k _ { S } = k _ { G \Sigma } + k _ { L D }
    \end{cases}
\]

电力系统的单位调节功率$k_S \quad$单位:$MW/Hz$ 或 $kW/Hz$

系统的单位调节功率越大,负荷增加所引起的频率变化越小,频率越稳定。

进行二次调频时系统的单位调节功率并未改变。二次调频使发电机的输出功率增加一个量,即对于同样的频率偏移,允许的负荷变化加大了(对于同样的负荷变化,系统的频率偏移将减小。)

一次调频:依靠发电机组调速器自动调节:只能实现有差调频;除满负荷运行的机组外,系统中的所有机组都参与。对应增大了的负荷,调速器调整的结果使发电机组输出功率增加,频率低于初始值;反之,如果负荷减小,调速器调整的结果使发电机组输出功率减小,频率高于初始值。

这种调节特性称为发电机原动机调速器的有差特性。有差特性的调速器,对发电机备用容量的要求相对较低,易于实现。而且可以避免发电机由于预留了较多的备用容量,而使发电机的利用率不高。

二次调频:手动或电动发电机组的调频器(同步器)来调节其有功功率输出的过程,可以实现频率的无差调节。效果就是平行移动功频静态特性。在一次调频的基础上,由一个或数个发电厂来承担。为了保证频率在额定值所允许的偏移范围内,电力系统运行中发电机组发出的有功功率必须和负荷消耗的有功功率在额定频率下平衡。

系统有功功率平衡方程式

有功功率平衡发电机组有功输出=所有负荷有功之和+电力网有功损耗之和+发电厂厂用电有功之和。
\[\sum P _ { G } = \sum P _ { L D } +\sum {\Delta P } +\sum P _ { p } + \sum P _ { re }\]

$\sum P _ { p }\quad$厂用电功率,$\sum P _ { re }\quad$备用功率

电力系统频率主要靠主调频厂负责调整,主调频厂选择的好坏直接关系到频率的质量。
主调频厂一般按下列条件选择:具有足够的调节容量和范围;具有较快的调节速度;具有安全性和经济性;机组调整功率时,在联络线上引起的功率变动及指定中枢点的电压波动应在允许范围内。
\subsection{电力系统的电压与无功功率}
中枢点调压

调整方式:

顺调压:最大负荷运行方式:低电压(中枢点的电压不应低于线路额定电压的102.5\%);
最小负荷运行方式:高电压(中枢点的电压不应高于线路额定电压的107.5\%)。
适用:中枢点到各负荷点近,负荷变动小,是调压要求最低的方式顺调压。


逆调压:最大负荷运行方式:高电压(中枢点的电压要比线路额定电压高5\%);
最小负荷运行方式:低电压(中枢点的电压要等于线路额定电压)。
适用:中枢点到各负荷点远,负荷变化较大,是调压要求较高的调压方式。

恒调压:适用介于上述两者之间,通常用于所供负荷波动较小的中枢点,最大和最小负荷运行时,保持中枢点电压在线路额定电压的102\%~105\%之间。

电压调整的措施调压措施有:改变发电机机端电压,需要改变发电机励磁电流;
改变升、降压变压器变比,需要调节变压器的分接头;改变网络无功功率分布,需要并联无功补偿装置;改变网路参数,需要线路串联电容补偿
\subsection{电力系统经济运行}
电网的能量损耗率,在同一时间内,电网损耗电量占供电 量的百分比,称为电网的损耗率,简称网损率或线损率,即:电力网损耗率=电力网损耗电量/供电量$\times$100\%

电力网的损耗电量:所有送电、变电和配电环节所损耗的电量。

供电量:在给定的时间内,系统中所有发电厂的总发电量同厂用电量之差。

能量损耗的近似计算方法:
1.最大负荷损耗时间法(针对电网规划阶段);
2.等值功率法(针对运行电网)。

降低网损的常用技术措施:
1.提高功率因数,减少网络中无功功率的传送;
2.合理组织或调整电力网的运行方式。(1)合理确定电力网的电压水平; 2)组织变压器的经济运行;3)闭式网实行功率的经济分布; 4)对电网进行技术改造。)

等耗量微增率准则:电力系统中的各发电机组按相等的耗量微增率运行,从而使得总的能源损耗最小,运行最经济。
\newpage{}
\section{第六章{} 电力系统的对称故障分析}
电磁暂态过程分析

简单故障:电力系统的某处只发生一种故障的情况。
\subsection{短路一般概念}
短路就是指相与相或相与地(对于中性点接地的系统)之间发生不正常通路的情況。

短路的种类

对称短路:三相短路5\%

不对称短路:单相接地短路65\%,两相短路10\%,两相接地短路20\%
\subsection{标幺值}
标幺制:采用物理量的标幺值进行分析计算的体制。
\[
    \text{标幺值}=\dfrac{\text{实际有名值(任何单位)}}{\text{基准值(与实际有名值同单位)}}
\]

(1)基准值选择:

基准值可以任选,但是为了简化电力系统分析计算,有默认的取值习惯!
基准功率多选为100MVA、1000MVA或系统总容量或某发电厂机组容量之和;基准电压为平均额定电压或额定电压。其次根据所选基准功率和基准电压确定基准电流和基准阻抗。

(2)不同设备的标么值计算方法:近似计算法

引入“平均额定电压”的定义

$U_{av}$:某级电网两侧变压器额定电压平均值

电压基准:$U_B$=$U_{av}$(平均额定电压)
我国常用的电力网额定电压与平均额定电压(单位:$kV$):
\[
    \begin{tabular}{ |c |c |c|c|c|c|c|c|c| }
        \hline
        $U$&3&6&10&35&110&220&330&500\\
        \hline
        $U_{av}$&3.15&6.3&10.5&37&115&230&345&525\\
        \hline
    \end{tabular}
\]

“近似”原则:
1.除电抗器外,假定同一电压等级中各元件的额定电 压等于$U_{av}$;
2.变压器的实际变比等于其两侧的$U_{av}$之比; 
3.基准电压取为$U_{av}$
\subsection{恒定电势源供电系统的三相短路}
恒定电势源:不论电力系统中发生什么扰动,电源的电压幅值和频率均保持恒定的电源。

三相短路时,短路电流的周期分量是三相对称的,非周期分量是三相不对称的,因而,非周期分量有最大初始值或零值的情况只可能在一相出现。
\[i _ { k a } = - I _ { p m } \operatorname { cos } \omega t + I _ { p m } e ^ { - \frac { t } { T _ { a } } }\]

短路电流的最大瞬时值约在短路后的T/2(T为工频分量的变化周期)时刻出现
\[t = \frac { T } { 2 } = \frac { \pi } { \omega } = 0.01 s\]

短路冲击电流:短路电流最大可能的瞬时值。
\[
    i_{im}=I_{pm}+I_{pm}e^{-0.01/T_a}=(1+e^{-{0.01/T_a}})I_{pm}=k_{im}I_{pm}
\]

其中,$k_{im}$为冲击系数,当在发电机端部短路时,$k_{im}$取1.9;当在发电厂高压侧母线上短路时,$k_{im}$取1.85;在其他点短路时,$k_{im}$取1.8。

短路电流有效值:以任一时刻t为中心的一个周期内瞬时电流的均方根值。

短路电流最大有效值:(出现在短路后第一个周期)
\[I_{im}=\sqrt{I_pt^2+I_{apt}^2}=I_p^2\sqrt{1+2(k_m-1)^2}\]

母线残压:三相金属性短路时,电源侧距故障点电抗为X的任意母线上的电压。
\[V_k=\sqrt{3}I_k^{(3)}X \qquad V_{k^*}=I^{(3)}_{k^*}X_*\]

表征暂态短路过程的几个衡量指标:短路冲击电流、短路电流的有效值、母线残压。
\subsection{有限容量电源的三相短路}
转移电抗(转移阻抗)定义:电源与短路点之间直接相连的等值阻抗。
如果仅在$i$支路中加电动势$\dot{E}$,其他电源电动势均为零时,则这$\dot{E}$与在k支路中所产生的电流的比值就是$i$支路与k支路之间的转移电抗。

输入阻抗定义:所有电源合并为一个等效电源,电源至故障节点的等效阻抗。在所有电源电动势均相等时,短路点的输入阻抗为其自导纳的倒数,其值等于短路点k对其余所有电源节点的转移电抗的并联值。

计算电抗:发电机的纵轴(直轴)次暂态(超瞬变)电抗和归算到发电机额定容量的外接电抗的标幺值之和。
\[X_{js}=X''_d+X_e\]

计算曲线法:通过计算电抗和在计算曲线或计算曲线数字表中查找任意时刻的短路电流。

发电机组合并和等值原则:%电源合并和化简的原则:

根据不同的具体条件,将网络中的电源分成几组,每组都用一个等值发电机来代表。

把短路电流变化规律大体相同的发电机尽可能多地合并起来。

对于条件比较特殊的某些发电机给以个别的考虑。

是否容许合并发电机的主要依据是:短路电流变化规律是否相同或相近。

主要的影响因素有两个:一个是发电机的特性(指类型和参数等);另一个是对短路点的电气距离。

在离短路点甚近时,发电机本身特性的不同对短路电流的变化规律具有决定性的影响。

如果短路点非常遥远,发电机到短路点之间的电抗数值甚大,发电机的参数不同对所引起的短路电流变化规律的差异影响较小。

电源合并原则:

将多个电源合并为一个等值电源:

使用前提:认为网络中所有发电机(不管电机型式和距短路点的远近等)在短路暂态过程中具有完全相同的变化规律。

步骤:认为发电机支路始端电位相同,用戴维南定理进行等值变换,等值电源容量取供给短路电流的所有电源额定容量之和,把等值总电抗归算为以等值容量和平均额定电压为基准值的计算电抗后,通过查图(表)得到短路电流标幺值。

将多个电源合并为若干个不同的等值电源:

(1)短路点的远近(以电源到短路点间的阻抗来区分)是影响电流变化规律的关键因素。短路点近时,电流初值大,变化也大;短路点远时,电流初值小,变化也小。因此距短路点远近相差很大的电源,即使是同类型的发电机也不能合并。
(2)发电机的类型也是影响电流变化规律的因素。短路点越近,发电机的类型对电流变化的影响越大,因此距短路点很近的不同类型的发电机是不能合并的。距短路点较远时,这种因发电机类型不同而引起的电 流差异就会减小,此时距短路点较远的不同类型的发电机是可以合并的。
(3)无限大容量电源不能合并,必须单独计算,因为它的短路电流在暂态过程中是不变的。
\newpage
\section{第七章{} 电力系统元件的序阻抗和等值网络}
\subsection{对称分量的原理}
三相参数对称的线性电路中,各相对称电流、电压之间的关系可以直接解耦,各相分量具有独立性。
三相不对称量可以是电压、电流、磁链等,为简便起见,以下的分析均以电压量的形式描述。
在电机学中已学过,一组三相不对称的电压可以由三组序电压分量来表示,各序电压、电流之间可以直接解耦,各序分量具有独立性,可以单独对各序分量进行分析计算。

序分量的独立性是对称分量法运算的前提。各序分量具有独立性的条件是三相线性电路参数对称。

实际复杂的电力系统通过各序网络等值化简(戴维南定理),仍然可以得到形式类似的方程式:
\[
    \begin{cases}
        \dot{E}_{eq}-Z_{ff(1)}\dot{I}_{fa(1)}=\dot{V}_{fa(1)}\\
        0-Z_{ff(2)}\dot{I}_{fa(2)}=\dot{V}_{fa(2)}\\
        0-Z_{ff(0)}\dot{I}_{fa(0)}=\dot{V}_{fa(0)}
    \end{cases}
\]
\subsection{变压器载各序电压作用下的等效电路机器序阻抗特性}
变压器各序等值电阻相等;
变压器各序等值漏抗相等。
负序励磁电抗=正序励磁电抗(取决于主磁通路径的磁阻,和电流序别特点及铁心结构有关

正、负序等值电路完全相同,零序等值电路与变压器铁芯结构、绕组连接方式及中性点工作方式有关。

变压器零序励磁电抗与变压器的铁心结构(磁路)密切相关。

变压器铁心结构的形式: 1三相组式;2三相四柱(或五柱式)式;3三相三柱式
\[X \propto L = \frac { N ^ { 2 } } { R _ { m } }\]

三个单相的组式,三相四柱式:磁阻很小,零序励磁电抗数值很大,等值电路中近似认为$X_{m0}$≈∞。可将励磁支路当做断开处理

三相三柱式变压器的磁阻很大,零序励磁电抗数值很小,励磁回路不能当做开路处理。(三相三柱式变压器不能忽略励磁支路)

只有当变压器的一次侧接触星形,其中性点直接接地(YN接法)或经阻抗接地时,在该侧绕组上加一组零序电动势后,绕组中才有可能有零序电流流过,才会呈现出一定数值的零序阻抗。在三绕组变压器中,为了消除三次谐波磁通的影响,使变压器的电动势接近正弦波,一般总有一个绕组连接成三角形。三角形接法的绕组能在变压器内部为零序电流的流通路径提供闭合回路,但不能为之相连的外电路提供零序电流回路,所以在零序等效电路中应将三角形接法的绕组短接,并将等效电路与外电路的连接隔断。零序电流即不能流入三角形接法的三相系统,也不能流出三角形接法的三相系统。只可以在三相绕组内部形成零序环流。

电力系统各序网络的制定

序网络的制定:

一、正序网络的制订

正序网络可参照三相短路计算时的网络进行分析,但需注意:
1.故障点的电压不为零而为正序电压;
2.中性点接地阻抗、空载线路(不计导纳)以及空载变压器(不计励磁支路)中不会有正序电流通过,不出现在正序网络中。
当特别说明负荷为大容量电动机或专门给出了负荷参数,在画正序网络时,需考虑大容量电动机会向电网反送短路电流,即所以负荷应用等值电动势和等值电抗表示。

二、负序网络的制定

1.组成负序网络的元件与组成正序网络的元件完全相同。
2.负序网络中发电机的电势为零。
3.元件参数取为负序电抗。各元件的负序电抗值,除发电机等旋转元件外,其他均与正序网络的相同。

三、零序网络的制订

制订零序网络时首先要查明在故障点加零序电压后零序电流可能通过的路径:将零序电流能通过的元件以其相应的零序电抗代替,即可组成零序网络。
1.组成零序网络的元件和组成正序、负序网络的元件不同。
2.从故障点开始,查明零序电流可能通过的路径。

负荷一般不需要建立零序等值电路(异步电动机三相绕组通常接成三角形或不接地星形)。
零序等值电路中电动机:主要看有没有专门说明电动机绕组的连接方法,如果说是星形且中性点接地,就和发电机一样处理:如果没说就按照是三角形或不接地星形接法处理,不出现在零序网。
\newpage{}
\section{第八章{} 电力系统不对称故障分析}
\subsection{简单不对称短路的分析}
不对称短路计算步骤:特殊相为基准相,首先计算基准相正序电流,再计算短路点各序电流、电压分量;然后根据需要计算各序电流、电压在网络中的分布;最后将各序分量合成可到网络各支路中各相电流和各节点上的各相电压。

如果故障是不接地性质(两相短路),则序分量表达式中不会出现零序阻抗,而且零序电流和零序电压必为零值;只有当故障具有接地性质(单相接地和两相接地)时,才会出现零序电流和零序电压,而其它序分量则均将与零序阻抗相关。

复合序网络法:按序量边界条件,将正序、负序和零序网络按一定规律连接起来,求出短路点各序电流和各序电压的方法。

一、单相接地短路

a相经附加电抗$X_f$接地短路

故障点的相量边界条件
\[ \dot { V } _ { f _ { a } } = j \dot { I } _ { f _ { a } } X _ { f }\quad \dot { I } _ { f b } = 0\quad \dot { I } _ { f c } = 0 \]

二、单相接地短路

a相直接(金属性)接地短路($X_f=0$)

故障点的相量边界条件
\[ \dot { V } _ { f _ { a } } = j \dot { I } _ { f _ { a } } X _ { f } = 0\quad \dot { I } _ { f b } = 0\quad \dot { I } _ { f c } = 0 \]

三、两相短路

假设b、c相经附加电抗$X_f$发生短路

故障点的相量边界条件
\[ \dot { I } _ { f a } = 0\quad \dot { I } _ { f b } +     \dot { I } _ { f c } = 0\quad \dot { V } _ { f _ { b } } - \dot { V } _ { f _ { c } } = j \dot { I } _ { f _ { b } } X _ { f } \]

四、两相短路接地

假设b、c相经附加电抗$X_f$发生短路接地
故障点的相量边界条件
\[ \dot { I } _ { f a } = 0\quad \dot { V } _ { f _ { b } } = \dot { V } _ { f _ { c } } = j ( \dot { I } _ { f _ { b } } + \dot { I } _ { f _ { c } } ) X _ { f } \]

各序电流和电压分布的计算
\[I_{120}=SI_{abc}\]
\[
    S=
    \frac13\begin{bmatrix}
        1 &  \alpha & \alpha ^2     \\
        1&  \alpha ^2&\alpha \\
        1&  1& 1
    \end{bmatrix} 
\]

对称分量法,结合故障点的相量边界条件求序分量边界条件
\[
    \begin{bmatrix}
        \dot {I}_{fa(1)}\\
        \dot {I}_{fa(2)}\\
        \dot {I}_{fa(0)}
    \end{bmatrix}=
    \frac13\begin{bmatrix}
        1 &  \alpha & \alpha ^2\\
        1&  \alpha ^2&\alpha \\
        1&  1& 1
    \end{bmatrix} 
    \begin{bmatrix}
        \dot {I}_{fa}\\
        \dot {I}_{fb} \\
        \dot {I}_{fc}
    \end{bmatrix}
\]

复合序网法关键步骤(以下依次对应)


单相接地短路

a相经附加电抗$X_f$接地短路
\[ \dot { I } _ { f a } ( 1 ) = \frac { \dot{E} _ { e q } } { j ( X _ { f ( 1 ) } +  X _ { f ( 2 ) } +  X _ { f ( 0 ) } +3 X _ {  f  } ) }\]

单相接地短路

a相直接(金属性)接地短路($X_f=0$)
\[ \dot { I } _ { f a } ( 1 ) = \frac {  \dot{E} _ { e q } } { j ( X _ { f ( 1 ) } +  X _ { f ( 2 ) }  + X _{f(0)} )  }\]

两相短路

假设b、c相经附加电抗$X_f$发生短路
\[ \dot { I } _ { f a } ( 1 ) = \frac {  \dot{E} _ { e q } } { j ( X _ { f ( 1 ) } +  X _ { f ( 2 ) }  + X _{f} )  }\]

两相短路接地

假设b、c相经附加电抗$X_f$发生短路接地
\[ \dot { I } _ { f a } ( 1 ) = \frac { \dot{E} _ { e q } } { j [ X _ { f ( 1 ) } +\frac{  X _ { f ( 2 ) } (  X _ { f ( 0 ) } +3 X _f )}{  X _ { f ( 2 ) } +  X _ { f ( 0 ) } +3 X _f} ] }\]
\[
    \begin{cases}
        \dot{V}_{fa(1)}  =  \dot{E}_{eq}-Z_{ff(1)}\dot{I}_{fa(1)}\\
        \dot{V}_{fa(2)}    = 0-Z_{ff(2)}\dot{I}_{fa(2)}\\
        \dot{V}_{fa(0)} =   0-Z_{ff(0)}\dot{I}_{fa(0)}
    \end{cases}
\]
\[
    \begin{bmatrix}
        \dot {V}_{fa}\\
        \dot {V}_{fb} \\
        \dot {V}_{fc}
    \end{bmatrix}=
    \begin{bmatrix}
        1&  1  & 1 \\
        \alpha ^2&\alpha & 1   \\
        \alpha & \alpha ^2& 1
    \end{bmatrix} 
    \begin{bmatrix}
        \dot {V}_{fa(1)}\\
        \dot {V}_{fa(2)}\\
        \dot {V}_{fa(0)}
    \end{bmatrix}
\]

正序等效定则表明,不对称短路时,短路点的正序电流大小与在短路点串接一与短路类型有关的附加电抗后发生三相短路时的电流相等,这一概念在仅对正序分量感兴趣的工程计算中很有作用。
\subsection{不对称短路时网络中电流和电压的分布}
正序电压在电源处最高,随着与短路点的接近而逐渐降低,在短路点处降到最低值。

负序电压在短路点处最高,随着与短路点的距离的增加而降低,在电源点处降到零。

零序电压在短路点处最高,随着与短路点的距离的增加而降低,在变压器三角形出线处降到零。

计算网络中任意支路的电流和节点电压的各序分量

计算方法;基尔霍夫电流和电压定律
\[\dot { V } _ { m ( 1 ) } = \dot { V } _ { fa( 1 ) } +   j \dot { I } _ { fa( 1 ) }X _ { l ( 1 ) }\]
\[\dot { V } _ { m ( 2 ) } = \dot { V } _ { fa( 2 ) } +   j \dot { I } _ { fa( 2 ) }X _ { l ( 2 ) }\]
\[\dot { V } _ { m ( 0 ) } = \dot { V } _ { fa( 0 ) } +   j \dot { I } _ { fa( 0 ) }X _ { l ( 0 ) }\]
\subsection{电流和电压各序分量经变压器后的相位变化}
电流经Y/d11连接的变压器后,正序分量相位将超前30度,而负序分量相位将落后30度,零序分量相位无变化。
\[\dot { V } _ { a ( 1 ) } = \dot { V } _ { A ( 1 ) } e ^ { j 30 ^ { \circ } } \qquad \dot{I} _ { a ( 1 ) } = \dot { I } _ { A ( 1 ) } e ^ {j30 ^ { \circ } }\]
\[\dot { V } _ { a ( 2 ) } = \dot { V } _ { A ( 2 ) } e ^ { -j 30 ^ { \circ } } \qquad\dot{I }_ { a ( 2 ) } = \dot { I } _ { A ( 2 ) } e ^ {-j30 ^ { \circ } }\]
\[\begin{cases}
    \dot { I } _ { a } = I _ { a ( 1 ) } + \dot { I } _ { a ( 2 ) } = \dot { I } _ { A (1) } e ^ { j30 ^ { \circ } } + \dot { I } _ { A ( 2 ) } e ^ { - j30 ^ { \circ } }\\
    \dot { I } _ { b } =a^2 \dot {I} _ { a ( 1 ) } + a \dot { I } _ { a ( 2 ) } =a^2  \dot { I } _ { A (1) } e ^ { j30 ^ { \circ } } + a \dot { I } _ { A ( 2 ) } e ^ { - j30 ^ { \circ } }\\
    \dot { I } _ { c } = a \dot {I} _ { a ( 1 ) } + a^2 \dot { I } _ { a ( 2 ) } =a  \dot { I } _ { A (1) } e ^ { j30 ^ { \circ } } + a^2 \dot { I } _ { A ( 2 ) } e ^ { - j30 ^ { \circ } } 
\end{cases}
\]
\[
    \begin{cases}
        \dot { V } _ { n ( 1 ) } = [ \dot { V } _ { f _ { a ( 1 ) } } + j\dot { I} _ { f _ { a ( 1 ) } }( X _ { l( 1 ) } + X _ {T ( 1 ) } ) ] e ^ { j30  ^ { \circ }}\\
        \dot { V } _ { n ( 2 ) } = [ \dot { V } _ { f _ { a ( 2 ) } } + j\dot { I} _ { f _ { a ( 2 ) } }( X _ { l( 2 ) } + X _ {T ( 2 ) } ) ] e ^ { -j30  ^ { \circ }}\\
        \dot { V } _ {n(0)} = 0    
    \end{cases}
\]
\newpage{}
\section{第九章{} 电力系统稳定的基本概念}
电力系统的稳定问题

根据电力系统失稳动态过程的特征,电力系统稳定可分为功角稳定、电压稳定和频率稳定三大类。

功角稳定又分为静态稳定、小干扰动态稳定、暂态稳定和大干扰动态稳定四个子类。功角稳定主要与有功功率平衡有关,其主要标志为同步发电机组之间的同步运行状态是否被破坏。功角δ既是两个发电机电势间的相位差,又是用电气角度表示的发电机转子之间的相对位移角(位置角)。

同步发电机的机电特性由同步发电机转子的运动特性
(转子运动方程:
$\frac{d^{2} \delta }{dt^{2}}=M_T-M_e$)
及发电机的电磁功率的变化特性
(功角特性:
$P=\frac{EU}{X_{\sum{}}}\sin{\delta}$)
共同决定。

静态稳定:在小扰动作用下,系统运行状态将有小变化而偏离原来的运行状态,如干扰不消失,系统能在偏离原水平衡点很小处建立新的平衡点,或当扰动消失后,系统能回到原有的平衡点,则称电力系统是静态稳定的。

暂态稳定:电力系统能正常工作的情况下,受到一个较大的扰动后,能从原来的运行状态过渡到新的运行状态,并能在新的运行状态下稳定地工作。

电压稳定性问题:
电压稳定主要与无功功率的平衡相关,系统受到干扰后因为无功功率供应不足或分布不合理,是否会因为某些节点电压严重偏离平衡状态而使系统失去稳定。其主要表现为系统的局部或全局是否发生电压的持续下降或上升。功角不稳定与电压不稳定二者相互区别,又相互联系。功角失稳有可能引发电压失稳,反之亦然。
\end{document}
